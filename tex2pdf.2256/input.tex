\documentclass[]{article}
\usepackage{lmodern}
\usepackage{amssymb,amsmath}
\usepackage{ifxetex,ifluatex}
\usepackage{fixltx2e} % provides \textsubscript
\ifnum 0\ifxetex 1\fi\ifluatex 1\fi=0 % if pdftex
  \usepackage[T1]{fontenc}
  \usepackage[utf8]{inputenc}
\else % if luatex or xelatex
  \ifxetex
    \usepackage{mathspec}
    \usepackage{xltxtra,xunicode}
  \else
    \usepackage{fontspec}
  \fi
  \defaultfontfeatures{Mapping=tex-text,Scale=MatchLowercase}
  \newcommand{\euro}{€}
\fi
% use upquote if available, for straight quotes in verbatim environments
\IfFileExists{upquote.sty}{\usepackage{upquote}}{}
% use microtype if available
\IfFileExists{microtype.sty}{%
\usepackage{microtype}
\UseMicrotypeSet[protrusion]{basicmath} % disable protrusion for tt fonts
}{}
\usepackage[margin=1in]{geometry}
\usepackage{color}
\usepackage{fancyvrb}
\newcommand{\VerbBar}{|}
\newcommand{\VERB}{\Verb[commandchars=\\\{\}]}
\DefineVerbatimEnvironment{Highlighting}{Verbatim}{commandchars=\\\{\}}
% Add ',fontsize=\small' for more characters per line
\usepackage{framed}
\definecolor{shadecolor}{RGB}{248,248,248}
\newenvironment{Shaded}{\begin{snugshade}}{\end{snugshade}}
\newcommand{\KeywordTok}[1]{\textcolor[rgb]{0.13,0.29,0.53}{\textbf{{#1}}}}
\newcommand{\DataTypeTok}[1]{\textcolor[rgb]{0.13,0.29,0.53}{{#1}}}
\newcommand{\DecValTok}[1]{\textcolor[rgb]{0.00,0.00,0.81}{{#1}}}
\newcommand{\BaseNTok}[1]{\textcolor[rgb]{0.00,0.00,0.81}{{#1}}}
\newcommand{\FloatTok}[1]{\textcolor[rgb]{0.00,0.00,0.81}{{#1}}}
\newcommand{\CharTok}[1]{\textcolor[rgb]{0.31,0.60,0.02}{{#1}}}
\newcommand{\StringTok}[1]{\textcolor[rgb]{0.31,0.60,0.02}{{#1}}}
\newcommand{\CommentTok}[1]{\textcolor[rgb]{0.56,0.35,0.01}{\textit{{#1}}}}
\newcommand{\OtherTok}[1]{\textcolor[rgb]{0.56,0.35,0.01}{{#1}}}
\newcommand{\AlertTok}[1]{\textcolor[rgb]{0.94,0.16,0.16}{{#1}}}
\newcommand{\FunctionTok}[1]{\textcolor[rgb]{0.00,0.00,0.00}{{#1}}}
\newcommand{\RegionMarkerTok}[1]{{#1}}
\newcommand{\ErrorTok}[1]{\textbf{{#1}}}
\newcommand{\NormalTok}[1]{{#1}}
\usepackage{longtable,booktabs}
\usepackage{graphicx}
\makeatletter
\def\maxwidth{\ifdim\Gin@nat@width>\linewidth\linewidth\else\Gin@nat@width\fi}
\def\maxheight{\ifdim\Gin@nat@height>\textheight\textheight\else\Gin@nat@height\fi}
\makeatother
% Scale images if necessary, so that they will not overflow the page
% margins by default, and it is still possible to overwrite the defaults
% using explicit options in \includegraphics[width, height, ...]{}
\setkeys{Gin}{width=\maxwidth,height=\maxheight,keepaspectratio}
\ifxetex
  \usepackage[setpagesize=false, % page size defined by xetex
              unicode=false, % unicode breaks when used with xetex
              xetex]{hyperref}
\else
  \usepackage[unicode=true]{hyperref}
\fi
\hypersetup{breaklinks=true,
            bookmarks=true,
            pdfauthor={Anamarija Mijatovic},
            pdftitle={Analiza gospodarskega pomena turizma in analiza najbolj obiskanih drzav},
            colorlinks=true,
            citecolor=blue,
            urlcolor=blue,
            linkcolor=magenta,
            pdfborder={0 0 0}}
\urlstyle{same}  % don't use monospace font for urls
\setlength{\parindent}{0pt}
\setlength{\parskip}{6pt plus 2pt minus 1pt}
\setlength{\emergencystretch}{3em}  % prevent overfull lines
\setcounter{secnumdepth}{0}

%%% Use protect on footnotes to avoid problems with footnotes in titles
\let\rmarkdownfootnote\footnote%
\def\footnote{\protect\rmarkdownfootnote}

%%% Change title format to be more compact
\usepackage{titling}

% Create subtitle command for use in maketitle
\newcommand{\subtitle}[1]{
  \posttitle{
    \begin{center}\large#1\end{center}
    }
}

\setlength{\droptitle}{-2em}
  \title{Analiza gospodarskega pomena turizma in analiza najbolj obiskanih drzav}
  \pretitle{\vspace{\droptitle}\centering\huge}
  \posttitle{\par}
  \author{Anamarija Mijatovic}
  \preauthor{\centering\large\emph}
  \postauthor{\par}
  \date{}
  \predate{}\postdate{}

\usepackage[slovene]{babel}
\usepackage{graphicx}


\begin{document}

\maketitle


\section{Analiza gospodarskega pomena turizma in analiza najbolj
obiskanih
držav}\label{analiza-gospodarskega-pomena-turizma-in-analiza-najbolj-obiskanih-drzav}

Analizirala bom celotni doprinos turizma BDP-ju in zaposlenosti za
posamezne države. Tako bom izvedela katera država je najbolj odvisna od
turizma. Prav tako pa bom analizirala vhodni turizem za posamezne
države, torej kolikšno število ljudi je obiskalo posamezno državo in s
tem bom izvedela katera država je najbolj obiskana.

Podatke sem dobila na:

\begin{itemize}
\itemsep1pt\parskip0pt\parsep0pt
\item
  \url{http://knoema.com/WTTC2015/world-travel-and-tourism-council-data-2015}
\item
  \url{http://knoema.com/TOURISM_INBOUND/inbound-tourism}
\item
  \url{http://data.worldbank.org/indicator/ST.INT.ARVL}
\end{itemize}

\begin{center}\rule{0.5\linewidth}{\linethickness}\end{center}

\begin{center}\rule{0.5\linewidth}{\linethickness}\end{center}

\section{Obdelava, uvoz in čiščenje
podatkov}\label{obdelava-uvoz-in-ciscenje-podatkov}

Uvozila sem podatke o turizmu za različne države s spletnih strani
Knoema.com in WorldBank. Tabelo, ki prikazuje doprinos BDP-ju in tabelo,
ki prikazuje doprinos turizmu sem uvozila kot CSV. Druga tabela nam
prikazuje število turistov, ki so obiskali posamezno državo v letih
2012,2013 in 2014. Podatke za 2014 sem uvozila kot CSV, za leti 2012 in
2013 pa kot HTML s spletne strani WorldBank-a.

Iz uvoženih tabel sem izbrisala nepotrebne stolpce, stolpce preimenovala
v slovenščino, spremenila vrednosti praznih nizov v NA, tako da sm na
koncu dobile tri urejene tabele, ki imajo takšne stolpce:

Prva tabela:

\begin{itemize}
\itemsep1pt\parskip0pt\parsep0pt
\item
  Drzava
\item
  enota (vrednosti so izražene v procentih ali US\$ bn)
\item
  Leto (podatki so za leta 2012,2013,2014)
\item
  Direktni doprinos BDP-ju
\item
  Totalni doprinos BDP-ju
\item
  Delez investicij v turizem
\item
  Potrosnja turistov
\end{itemize}

Druga tabela:

\begin{itemize}
\itemsep1pt\parskip0pt\parsep0pt
\item
  Drzava
\item
  enota (vrednosti so izražene v procentih ali tisočih)
\item
  Leto (podatki so za leta 2012,2013,2014)
\item
  Direkten doprinos zaposlenosti
\item
  Totalni doprinos zaposlenosti
\end{itemize}

Tretja tabela:

\begin{itemize}
\itemsep1pt\parskip0pt\parsep0pt
\item
  Drzava
\item
  Leto (podatki so za leta 2012,2013,2014)
\item
  Stevilo turistov
\end{itemize}

Oglejmo si kako izgledajo tabele:

\begin{Shaded}
\begin{Highlighting}[]
\KeywordTok{kable}\NormalTok{(}\KeywordTok{head}\NormalTok{(tabela))}
\end{Highlighting}
\end{Shaded}

\begin{longtable}[c]{@{}lllrrrr@{}}
\toprule
Drzava & enota & Leto & Direktni doprinos BDP-ju & Totalni doprinos
BDP-ju & Delez investicij v turizem & Potrosnja turistov\tabularnewline
\midrule
\endhead
Slovenia & \% share & 2012 & 3.500 & 12.800 & 13.000 &
7.800\tabularnewline
Slovenia & \% share & 2013 & 3.600 & 13.200 & 13.600 &
8.100\tabularnewline
Slovenia & \% share & 2014 & 3.700 & 13.600 & 13.900 &
8.300\tabularnewline
Slovenia & US\$ bn & 2012 & 1.638 & 5.967 & 1.051 & 2.699\tabularnewline
Slovenia & US\$ bn & 2013 & 1.690 & 6.116 & 1.063 & 2.843\tabularnewline
Slovenia & US\$ bn & 2014 & 1.713 & 6.187 & 1.086 & 2.938\tabularnewline
\bottomrule
\end{longtable}

\begin{Shaded}
\begin{Highlighting}[]
\KeywordTok{kable}\NormalTok{(}\KeywordTok{head}\NormalTok{(zaposlenost))}
\end{Highlighting}
\end{Shaded}

\begin{longtable}[c]{@{}lllrr@{}}
\toprule
Drzava & enota & Leto & Direkten doprinos zaposlenosti & Totalni
doprinos zaposlenosti\tabularnewline
\midrule
\endhead
Slovenia & \% share & 2012 & 3.9 & 13.1\tabularnewline
Slovenia & \% share & 2013 & 4.0 & 13.5\tabularnewline
Slovenia & \% share & 2014 & 4.2 & 13.9\tabularnewline
Slovenia & thousands & 2012 & 31.5 & 105.7\tabularnewline
Slovenia & thousands & 2013 & 32.5 & 108.1\tabularnewline
Slovenia & thousands & 2014 & 33.7 & 111.3\tabularnewline
\bottomrule
\end{longtable}

\begin{Shaded}
\begin{Highlighting}[]
\KeywordTok{kable}\NormalTok{(}\KeywordTok{head}\NormalTok{(vstop2))}
\end{Highlighting}
\end{Shaded}

\begin{longtable}[c]{@{}llr@{}}
\toprule
Drzava & Leto & Stevilo turistov\tabularnewline
\midrule
\endhead
Armenia & 2012 & 963000\tabularnewline
Australia & 2012 & 6032000\tabularnewline
Austria & 2012 & 24151000\tabularnewline
Azerbaijan & 2012 & 1986000\tabularnewline
Belarus & 2012 & 119000\tabularnewline
Belgium & 2012 & 7560000\tabularnewline
\bottomrule
\end{longtable}

\begin{center}\rule{0.5\linewidth}{\linethickness}\end{center}

\section{Analiza in vizualizacija
podatkov}\label{analiza-in-vizualizacija-podatkov}

\begin{verbatim}
## Regions defined for each Polygons
## Regions defined for each Polygons
\end{verbatim}

\includegraphics{projekt_files/figure-latex/zemljevid1-1.pdf}

Zemljevid prikazuje Evropo, z barvo je označen totalni doprinos BDP-ju
od turizma. Vidimo da največji delež ima Hrvaška, sledijo pa ji Črna
gora, Islandija, Španija in Grčija.

\textbackslash{}begin\{figure\}

\{\centering \includegraphics{projekt_files/figure-latex/graf1-1}

\}

\textbackslash{}caption\{Drzave ki imajo največji delez BDP-ja ki ga
predstavlja turizem (\%)\}\label{fig:graf1}
\textbackslash{}end\{figure\}

Iz grafa razberemo da Aruba, Antigva in Barbuda, Sejšeli, Vanatu in
Maldivi in imajo največji totalni doprinos BDP-ju in sicer vse države
imajo okoli 50\%. Opazimo da so to predvsem otoške države, za katere je
znažilno da so najbolj odvisne od turizma in da jim je turizem glavna
gospodarka aktivnost.

\begin{figure}

{\centering \includegraphics{projekt_files/figure-latex/graf2-1} 

}

\caption{Totalni doprinos BDP-ju}\label{fig:graf2}
\end{figure}

V tem grafu pa je prikazan direktni doprinos turizma BDP-ju. Vidimo da
so države iste kot v prejšnem grafu, na novo so le Bahami. Razlika med
direktnim in totalni doprinosom turizma je, da direktni doprinos
upošteva le dohodke, ki jih direktno porabijo turisti (za plačanje
hotelov, nakupi suvenirjev, plačilo v restevracijah,itd.) v državi.
Totalni doprinos pa upošteva še davke in dajatve na izdatke in storitve,
ki jih uporabljajo turisti. Totalni doprinos je bolj pomemben za
analiziranje odviisnosti države od turizma, saj vključuje vse prihodke s
strani turizma.

\begin{figure}

{\centering \includegraphics{projekt_files/figure-latex/graf3-1} 

}

\caption{Doprinos zaposlenosti}\label{fig:graf3}
\end{figure}

Graf nam prikazuje 5 držav, pri katerih je največji procent zaposlenih v
turizmu. Opazimo da so države iste kot v grafu, ki prikazuje največji
delež BDP-ja. To smo na nek način tudi pričakovali, saj je logično da
države, ki so najbolj odvisne od turizma in ``živijo'' od turizma, bode
imele največ zaposlenih v turizmu. Največji procent zaposlenih ima tako
kot pri BDP-ju Aruba.

\begin{figure}

{\centering \includegraphics{projekt_files/figure-latex/graf5-1} 

}

\caption{Drzave,ki najvec vlozijo v turizem}\label{fig:graf5}
\end{figure}

V grafu vidimo 5 držav, ki največjo količino denarja investirajo v
turizem. Med njimi sta daleč vodilni ZDA (leta 2014 je vložila kar
178.384 miljard\$ ) in Kitajska, ki je v letu 2014 investirala 133.159
miljard\$ v turizem. Iz tabele lahko razberemo da ZDA predstavlja
tolikšna vsota denarja le 6.3\% celotnih investicij, Kitajski pa 2.8\%.

\begin{figure}

{\centering \includegraphics{projekt_files/figure-latex/graf6-1} 

}

\caption{Drzave,v katerih turisti najvec trosijo}\label{fig:graf6}
\end{figure}

Iz grafa vidimo v katerih državah turisti zapravijo največ denarja, med
njimi daleč vodilne so ZDA.V ``TOP 5'' pa se še uvrščajo Kitajska,
Francija, Španija in Nemčija. V naslednjem zamljevidu bomo videli da so
to tudi najbolj obiskane države.

\includegraphics{projekt_files/figure-latex/zemljevid2-1.pdf}

Zemljevid je narisan za leto 2013. Opazimo da najbolj obiskane države so
bile Francija (z več kot 80 miljonov turistov), ZDA, Kitajska, Španija,
Nemčija. In še graf, v kateremu bomo bolj točno videli kolikšno število
ljudi vsakoletno obišče te države.

\begin{figure}

{\centering \includegraphics{projekt_files/figure-latex/graf7-1} 

}

\caption{Najvec turistov}\label{fig:graf7}
\end{figure}

Poglejmo si še podatke za Slovenijo:

\begin{Shaded}
\begin{Highlighting}[]
\KeywordTok{kable}\NormalTok{(Slovenija)}
\end{Highlighting}
\end{Shaded}

\begin{longtable}[c]{@{}lllrrrr@{}}
\toprule
Drzava & enota & Leto & Direktni doprinos BDP-ju & Totalni doprinos
BDP-ju & Delez investicij v turizem & Potrosnja turistov\tabularnewline
\midrule
\endhead
Slovenia & \% share & 2012 & 3.500 & 12.800 & 13.000 &
7.800\tabularnewline
Slovenia & \% share & 2013 & 3.600 & 13.200 & 13.600 &
8.100\tabularnewline
Slovenia & \% share & 2014 & 3.700 & 13.600 & 13.900 &
8.300\tabularnewline
Slovenia & US\$ bn & 2012 & 1.638 & 5.967 & 1.051 & 2.699\tabularnewline
Slovenia & US\$ bn & 2013 & 1.690 & 6.116 & 1.063 & 2.843\tabularnewline
Slovenia & US\$ bn & 2014 & 1.713 & 6.187 & 1.086 & 2.938\tabularnewline
\bottomrule
\end{longtable}

\begin{Shaded}
\begin{Highlighting}[]
\KeywordTok{kable}\NormalTok{(slozaposlenost)}
\end{Highlighting}
\end{Shaded}

\begin{longtable}[c]{@{}lllrr@{}}
\toprule
Drzava & enota & Leto & Direkten doprinos zaposlenosti & Totalni
doprinos zaposlenosti\tabularnewline
\midrule
\endhead
Slovenia & \% share & 2012 & 3.9 & 13.1\tabularnewline
Slovenia & \% share & 2013 & 4.0 & 13.5\tabularnewline
Slovenia & \% share & 2014 & 4.2 & 13.9\tabularnewline
Slovenia & thousands & 2012 & 31.5 & 105.7\tabularnewline
Slovenia & thousands & 2013 & 32.5 & 108.1\tabularnewline
Slovenia & thousands & 2014 & 33.7 & 111.3\tabularnewline
\bottomrule
\end{longtable}

V Sloveniji predstavlja turizem povprečno 3,5\% BDP-ja , skupni delež
turizma pa je približno 13\% BDP-ja.Turistična industrija zaposluje
povprečno 4.\% kar je 31500 ljudi,turizem v širšem smislu pa ustvari v
povprečju 108000 delovnih mest kar predstavlja 32\% od vseh delovnih
mest v Sloveniji. Slovenija povprečno investira 13.5\% v turizem in
sicer okoli 1 miljard \$ turisti pa v Sloveniji zapravijo okoli 2
miljardi dolarjev.Upoštevajoč vsa merila, se Slovenija uvršča na 76.
mesto po tem, kako pomemben je v gospodarskem smislu turizem. Glede
prispevka turizma v BDP pa na 63. mesto. V spodnji tabeli si lahko
ogledamo koliko turistov je obiskalo Slovenijo. V povprečju jo obišče
letno okoli 2 miljona turistov.

\begin{Shaded}
\begin{Highlighting}[]
\KeywordTok{kable}\NormalTok{(Slovenija_vstop)}
\end{Highlighting}
\end{Shaded}

\begin{longtable}[c]{@{}llr@{}}
\toprule
Drzava & Leto & Stevilo turistov\tabularnewline
\midrule
\endhead
Slovenia & 2012 & 2156000\tabularnewline
Slovenia & 2013 & 2259000\tabularnewline
Slovenia & 2014 & 2411000\tabularnewline
\bottomrule
\end{longtable}

\begin{center}\rule{0.5\linewidth}{\linethickness}\end{center}

\end{document}
